\documentclass[a4paper,12pt]{article}

\usepackage[ngerman]{babel}
\usepackage{amsmath, amssymb}
\usepackage{graphicx}
\usepackage{caption}
\usepackage{siunitx}
\usepackage{geometry}
\usepackage{float}
\usepackage{tikz}
\usepackage{booktabs}
\usepackage{hyperref}
\usepackage{csquotes}
\usepackage{pdfpages}
\usepackage{subcaption}


\usetikzlibrary{circuits.ee.IEC, positioning}

\geometry{margin=2.5cm}

\title{Wissenschaftliche Dokumentation \\[4mm]
       \Large Bestimmung der Ersatzparameter eines Dreileiterkabels mittels COMSOL}
\author{Laurens Perseus, Gian Marco Näf \\ Studiengang Elektrotechnik \\ Modul: Angewandter Elektromagnetismus}
\date{\today}

\begin{document}

\maketitle

\tableofcontents
\newpage

% ------------------------------------------------------------
\section*{Aufgabenstellung}
% ------------------------------------------------------------



\includepdf[pages={1}]{images/Aufgabenstellung.pdf}


\newpage

% ------------------------------------------------------------
\section{Einleitung}
% ------------------------------------------------------------

Die elektromagnetische Charakterisierung eines mehradrigen Kabels erfordert die
Bestimmung von Induktivitäten, Widerständen und Kapazitäten über geeignete
Maxwell-basierte Modelle. Die theoretischen Grundlagen folgen der klassischen
Elektrodynamik nach Jackson \cite{jackson_classical}, Griffiths
\cite{griffiths_em} und Sadiku \cite{sadiku_fields}. 

Alle numerischen Berechnungen werden in COMSOL Multiphysics durchgeführt,
entsprechend den Dokumentationen in \cite{comsol_reference}.

Modellparameter:

\begin{itemize}
    \item Abschirmung: $r_a = \SI{5}{mm}$, $r_i = \SI{4.5}{mm}$
    \item Leiter: Radius $r_l = \SI{1}{mm}$
    \item Dielektrikum: relative Permittivität $\varepsilon_r = 4$
    \item Leiterpositionen: $r_1=2.5$ mm, $r_2=2.3$ mm, $r_3=2.1$ mm
\end{itemize}

\newpage

% ------------------------------------------------------------
\section{Geometriemodell}
% ------------------------------------------------------------

Das Modell wurde als 2D-Querschnitt aufgebaut. Der Innenraum ist vollständig mit einem
homogenen Dielektrikum gefüllt. Eine äußere Luftdomäne wurde hinzugefügt, um die
Feldabnahme korrekt abzubilden. Gemäß \cite{comsol_reference} wird empfohlen,
die Luftdomäne auf mindestens das Fünffache des Außenradius zu setzen.

Zusätzlich wurde die maximale Elementskalierung (\textit{maximum element size})
auf den dimensionslosen Wert \textit{0.2} festgelegt, um eine ausreichende
Feldauflösung an den Leitern zu gewährleisten.

Der Radius der äußeren Luftdomäne wurde auf \SI{0.01}{m} gesetzt – ein optimaler
Kompromiss aus Genauigkeit und Rechenzeit.

\begin{figure}[H]
    \centering
    \includegraphics[width=0.6\textwidth]{images/geometry.png}
    \caption{Geometrie des Dreileiterkabels.}
\end{figure}

\newpage
% ------------------------------------------------------------
% Parametervariation der magnetischen Energie
% ------------------------------------------------------------
\section{Parametervariation der magnetischen Energie}

Zur Untersuchung des Einflusses der geometrischen Ausdehnung des Luftbereichs auf das magnetische Verhalten des Systems wurde eine parametische Studie durchgeführt. Dabei wurde der Luftradius $r_\mathrm{air}$ schrittweise von \SI{0.1}{\meter} bis \SI{2}{\meter} in Schritten von \SI{0.05}{\meter} variiert. Für jeden Parameterwert wurde eine magneto-quasistatische Simulation durchgeführt.

Als Auswertegröße dient die im Modell gespeicherte magnetische Energie $W_\mathrm{m}$, welche aus der magnetischen Energiedichte über das gesamte Rechengebiet integriert wurde. Diese Größe ist direkt mit der effektiven Induktivität des Systems verknüpft und eignet sich daher zur Beurteilung der Konvergenz sowie des Einflusses der äußeren Modellgrenzen.

\begin{figure}[H]
    \centering
    \includegraphics[width=0.7\textwidth]{images/magnetic_flux_density_r_l.png}
    \caption{Magnetische Energie $W_\mathrm{m}$ in Abhängigkeit vom Luftradius $r_\mathrm{air}$ aus einer parametischen Studie.}
    \label{fig:magnetic_energy_radius}
\end{figure}

Abbildung~\ref{fig:magnetic_energy_radius} zeigt den Verlauf der magnetischen Energie in Abhängigkeit vom Luftradius. Mit zunehmendem Radius vergrößert sich das feldführende Volumen, wodurch sich die magnetische Energie zunächst deutlich ändert. Ab einem bestimmten Radius geht der Kurvenverlauf in ein nahezu konstantes Plateau über. 

Dieses Verhalten zeigt, dass der Einfluss der äußeren Randbedingungen auf das magnetische Feld ab diesem Punkt vernachlässigbar ist. Der gewählte Luftradius kann somit als ausreichend groß betrachtet werden, um Randartefakte zu vermeiden. Gleichzeitig stellt dieser Radius einen sinnvollen Kompromiss zwischen numerischer Genauigkeit und Rechenaufwand dar.

\newpage
% ------------------------------------------------------------
\section{Aufgabe 1: Mesh-Studie}
% ------------------------------------------------------------

Zur Überprüfung der Netzkonvergenz wird die magnetische Energie $W_\mathrm{mag}$ betrachtet. Diese beschreibt die im Magnetfeld gespeicherte Energie und wird aus der magnetischen Flussdichte $\mathbf{B}$ und der magnetischen Feldstärke $\mathbf{H}$ über das gesamte Rechengebiet berechnet. Da diese Größe vom resultierenden Magnetfeld abhängt, eignet sie sich gut, um den Einfluss der Netzverfeinerung zu beurteilen.

Im Rahmen der Konvergenzstudie wurde das Netz schrittweise verfeinert. Für jede Netzstufe wurde eine magneto-quasistatische Simulation durchgeführt und anschließend die magnetische Energie ausgewertet. Durch den Vergleich der Ergebnisse kann festgestellt werden, wie stark sich die Lösung bei einer weiteren Verfeinerung des Netzes noch ändert.

Der in der Abbildung dargestellte Verlauf zeigt, dass sich die magnetische Energie mit zunehmender Netzfeinheit stabilisiert. Ab einer bestimmten Netzauflösung sind die Änderungen nur noch gering. Dies deutet darauf hin, dass das Ergebnis nicht mehr wesentlich vom Netz abhängt.

Das in Abbildung~\ref{fig_mesh_final} gezeigte Netz wurde daher als finales Netz gewählt. Es bietet eine ausreichende Genauigkeit bei gleichzeitig vertretbarem Rechenaufwand und wird für alle weiteren Simulationen verwendet.


\[
W_\mathrm{mag} = \frac{1}{2} \int \mathbf{B} \cdot \mathbf{H}\, dV
\]
wie definiert in \cite{jackson_classical}.  


\begin{figure}[H]
    \centering
    \includegraphics[width=0.65\textwidth]{images/final_mesh.png}
    \caption{Finales Netz nach Konvergenzstudie.}
    \label{fig_mesh_final}
\end{figure}

\begin{figure}[H]
    \centering
    \includegraphics[width=0.65\textwidth]{images/meshgraph.png}
    \caption{Konvergenzverlauf der magnetischen Energie.}
\end{figure}

\newpage

% ------------------------------------------------------------
\section{Aufgabe 2: Magneto-quasistatische Simulation}
% ------------------------------------------------------------
Die Induktivitäten der Leiter werden über das magnetische Vektorpotential bestimmt. Dabei beschreibt $L_{ij}$ die Selbst- bzw. Gegeninduktivität zwischen Leiter $i$ und Leiter $j$. Das magnetische Vektorpotential $\mathbf{A}_i$ wird durch den Strom im Leiter $i$ erzeugt, während $\mathbf{J}_j$ die Stromdichte im Leiter $j$ darstellt. Durch die Integration über das gesamte Volumen wird die magnetische Kopplung der beiden Leiter erfasst. Die Normierung mit dem Quadrat des Stroms stellt sicher, dass die Induktivität unabhängig von der gewählten Stromstärke ist.

Die Induktivität zweier Leiter ergibt sich aus dem magnetischen Vektorpotential
\cite{sadiku_fields}:

\[
L_{ij} = \frac{1}{I_j^2}\int \mathbf{A}_i \cdot \mathbf{J}_j\, dV .
\]

Die Leitungswiderstände ergeben sich über das Materialgesetz:

\[
R = \rho \frac{L}{A},
\]

siehe \cite{griffiths_em}.

Die elektrischen Widerstände der Leiter werden aus dem Materialgesetz berechnet. Der Widerstand $R$ hängt dabei vom spezifischen elektrischen Widerstand des Materials $\rho$, der Leiterlänge $L$ sowie der Querschnittsfläche $A$ ab. Ein längerer Leiter oder ein Material mit höherem spezifischem Widerstand führt zu einem größeren Widerstand, während ein größerer Leiterquerschnitt den Widerstand reduziert.

\subsection*{Ermittelte Induktivitäten und Widerstände}

Die aus COMSOL extrahierten Werte lauten:

\begin{table}[H]
\centering
\caption{Ermittelte Induktivitäten und Widerstände.}
\begin{tabular}{c|c}
\toprule
Parameter & Wert \\
\midrule
$L_{11}$ & $3.0933 \times 10^{-7}\,\mathrm{H}$ \\
$L_{12}$ & $3.1359 \times 10^{-7}\,\mathrm{H}$ \\
$L_{13}$ & $3.2381 \times 10^{-7}\,\mathrm{H}$ \\
$R_1$    & $6.7021 \times 10^{-5}\,\Omega$ \\
\bottomrule
\end{tabular}
\end{table}

Aus Symmetriegründen gilt:

\[
L_{12}=L_{21}, \qquad L_{13}=L_{31}.
\]

Weitere Einträge der Matrix ergeben sich analog durch Bestromung der anderen Leiter.

\subsection*{Magnetische Flussdichte}

\begin{figure}[H]
    \centering
    \includegraphics[width=0.65\textwidth]{images/magnetic_flux_density.png}
    \caption{Magnetische Flussdichte $|\mathbf{B}|$ für den Gesamtfall.}
\end{figure}

\subsection*{Magnetische Flussdichte – alle Einzelfälle}

\begin{figure}[H]
    \centering

    % --- Erste Zeile: B-Feld ---
    \begin{subfigure}[t]{0.32\textwidth}
        \centering
        \includegraphics[width=\linewidth]{images/B_case1.png}
        \caption{Leiter 1}
    \end{subfigure}
    \hfill
    \begin{subfigure}[t]{0.32\textwidth}
        \centering
        \includegraphics[width=\linewidth]{images/B_case2.png}
        \caption{Leiter 2}
    \end{subfigure}
    \hfill
    \begin{subfigure}[t]{0.32\textwidth}
        \centering
        \includegraphics[width=\linewidth]{images/B_case3.png}
        \caption{Leiter 3}
    \end{subfigure}

    \vspace{4mm}

    % --- Zweite Zeile: Stromdichte ---
    \begin{subfigure}[t]{0.32\textwidth}
        \centering
        \includegraphics[width=\linewidth]{images/J_case1.png}
        \caption{Leiter 1}
    \end{subfigure}
    \hfill
    \begin{subfigure}[t]{0.32\textwidth}
        \centering
        \includegraphics[width=\linewidth]{images/J_case2.png}
        \caption{Leiter 2}
    \end{subfigure}
    \hfill
    \begin{subfigure}[t]{0.32\textwidth}
        \centering
        \includegraphics[width=\linewidth]{images/J_case3.png}
        \caption{Leiter 3}
    \end{subfigure}

    \caption{Magnetische Flussdichte $|\mathbf{B}|$ (oben) und Stromdichte $|\mathbf{J}|$ (unten) für Einzellast der drei Leiter.}
\end{figure}


\newpage

% ------------------------------------------------------------
\section{Aufgabe 3: Elektrostatische Simulation}
% ------------------------------------------------------------

Die elektrostatische Feldverteilung im Modell wird durch das elektrische Skalarpotential $\phi$ beschrieben. Die Grundgleichung stellt sicher, dass in ladungsfreien Gebieten die Divergenz des elektrischen Flusses verschwindet. Das Materialverhalten geht dabei über die Permittivität $\varepsilon$ in die Gleichung ein und bestimmt die Feldverteilung im Isolationsmaterial.

Aus der berechneten elektrischen Flussdichte $\mathbf{D}$ kann die auf einem Leiter gespeicherte Ladung bestimmt werden. Dazu wird der Normalanteil der Flussdichte über die Oberfläche des jeweiligen Leiters integriert. Diese Ladung hängt von der angelegten elektrischen Spannung ab.

Durch Anlegen einer Spannung $V_j$ an einen Leiter und Berechnung der resultierenden Ladung $Q_i$ an allen Leitern kann die Kapazitätsmatrix bestimmt werden. Die Matrixelemente $C_{ij}$ beschreiben dabei sowohl die Eigenkapazitäten der einzelnen Leiter als auch die kapazitiven Kopplungen zwischen den Leitern.


Die elektrostatische Grundgleichung lautet \cite{griffiths_em}:

\[
\nabla \cdot (\varepsilon \nabla \phi) = 0.
\]

Die Ladung eines Leiters ergibt sich zu:

\[
Q_i = \int \mathbf{D}\cdot \mathbf{n}\, dA,
\]

woraus die Kapazitätsmatrix folgt:

\[
C_{ij} = \frac{Q_i}{V_j}.
\]

\subsection*{Potentialfeld}

\begin{figure}[H]
    \centering
    \includegraphics[width=0.65\textwidth]{images/electric_potential.png}
    \caption{Elektrisches Potentialfeld bei $V_1=1$\,V.}
\end{figure}

\subsection*{Elektrische Feldstärke}

\begin{figure}[H]
    \centering
    \includegraphics[width=0.65\textwidth]{images/electric_field.png}
    \caption{Elektrische Feldstärke $|\mathbf{E}|$.}
\end{figure}

\begin{table}[H]
\centering
\caption{Kapazitätsmatrix des Dreileiterkabels (in pF).}
\begin{tabular}{c|cccc}
 & Leiter 1 & Leiter 2 & Leiter 3 & Schirm \\
\hline
Leiter 1 & 86.875 & -46.347 & -3.1250 & -37.403 \\
Leiter 2 & -46.347 & 115.99 & -44.687 & -24.952 \\
Leiter 3 & -3.1250 & -44.687 & 79.474 & -31.663 \\
Schirm   & -37.403 & -24.952 & -31.663 & 94.017 \\
\end{tabular}
\end{table}

\newpage

% ------------------------------------------------------------
\section{Aufgabe 4: Pi-Ersatzschaltbild}
% ------------------------------------------------------------

\begin{figure}[H]
    \centering
    \includegraphics[width=0.8\textwidth]{images/pi_ersatzbild.png}
    \caption{Pi-Ersatzschaltbild des Dreileiterkabels.}
\end{figure}
\begin{table}[H]
\centering
\caption{Parameter des Pi-Ersatzschaltbilds des Dreileiterkabels.}
\begin{tabular}{c|c|c}
\toprule
Parameter & Beschreibung & Wert \\
\midrule
$R_1$ & Widerstand Leiter 1 & $6.70 \times 10^{-5}\,\Omega$ \\
$R_2$ & Widerstand Leiter 2 & $6.70 \times 10^{-5}\,\Omega$ \\
$R_3$ & Widerstand Leiter 3 & $6.70 \times 10^{-5}\,\Omega$ \\
\midrule
$L_1$ & Eigeninduktivität Leiter 1 & $3.09 \times 10^{-7}\,\mathrm{H}$ \\
$L_2$ & Eigeninduktivität Leiter 2 & $3.09 \times 10^{-7}\,\mathrm{H}$ \\
$L_3$ & Eigeninduktivität Leiter 3 & $3.09 \times 10^{-7}\,\mathrm{H}$ \\
\midrule
$C_1$ & Kapazität Leiter 1 – Schirm & $37.40\,\mathrm{pF}$ \\
$C_2$ & Kapazität Leiter 2 – Schirm & $24.95\,\mathrm{pF}$ \\
$C_3$ & Kapazität Leiter 3 – Schirm & $31.66\,\mathrm{pF}$ \\
\midrule
$C_4$ & Kapazität Leiter 1 – Leiter 2 & $46.35\,\mathrm{pF}$ \\
$C_5$ & Kapazität Leiter 1 – Leiter 3 & $3.13\,\mathrm{pF}$ \\
$C_6$ & Kapazität Leiter 2 – Leiter 3 & $44.69\,\mathrm{pF}$ \\
\bottomrule
\end{tabular}
\end{table}

Die Tabelle fasst alle für das Pi-Ersatzschaltbild relevanten Parameter zusammen.
Die Widerstände und Induktivitäten beschreiben die Längsparameter der Leiter,
während die Kapazitäten die Kopplungen der Leiter untereinander sowie gegen den
Schirm modellieren. Damit ist das Kabel vollständig für zeit- und
frequenzabhängige Systemsimulationen beschrieben.


\newpage

% ------------------------------------------------------------
\section{Fazit}
% ------------------------------------------------------------

Alle relevanten Ersatzparameter des Kabels wurden erfolgreich bestimmt.
Die Simulationsergebnisse zeigen die magnetischen Kopplungen zwischen den Leitern klar auf.
Durch die vollständige Bestimmung der Induktivitäts-, Kapazitäts- und Widerstandsparameter
kann das Kabel realitätsnah in Gesamtsystemsimulationen eingebettet werden.

\newpage

% ------------------------------------------------------------
\section*{Eidesstattliche Erklärung}
% ------------------------------------------------------------

Hiermit erklären wir, dass wir die vorliegende Arbeit selbstständig und ohne unerlaubte 
Hilfen erstellt haben. Alle verwendeten Quellen sind vollständig angegeben. Die Arbeit 
wurde zuvor nicht eingereicht.

\vspace{2cm}

Ort, Datum: \hrulefill

\vspace{1.5cm}

Unterschriften: \hrulefill

\newpage
\bibliographystyle{IEEEtran}
\bibliography{references}

\end{document}
