\documentclass[a4paper,12pt]{article}

\usepackage[ngerman]{babel}
\usepackage{amsmath, amssymb}
\usepackage{graphicx}
\usepackage{caption}
\usepackage{siunitx}
\usepackage{geometry}
\usepackage{float}
\usepackage{tikz}
\usetikzlibrary{circuits.ee.IEC, positioning}

\geometry{margin=2.5cm}

\title{Wissenschaftliche Dokumentation \\[4mm]
       \Large Bestimmung der Ersatzparameter eines Dreileiterkabels mittels COMSOL}
\author{Studiengang Elektrotechnik \\ Modul: Angewandter Elektromagnetismus}
\date{\today}

\begin{document}

\maketitle

\tableofcontents
\newpage

\section{Einleitung}

In dieser Dokumentation wird ein dreipoliges, abgeschirmtes Kabel elektromagnetisch 
untersucht. Ziel ist die Bestimmung der elektrischen Ersatzparameter für ein 
Pi-Ersatzschaltbild. Alle Simulationen werden in COMSOL Multiphysics durchgeführt.  
Die Aufgabenstellung folgt den Vorgaben aus der Simulationseinheit 
(Prof.~Dr. Smajic, OST).  

Die folgenden Parameter sind vorgegeben:
\begin{itemize}
    \item Abschirmung: äußerer Radius $r_a = \SI{5}{mm}$, innerer Radius $r_i = \SI{4.5}{mm}$
    \item Leiter: Radius $r_l = \SI{1}{mm}$, elektrisch leitfähig
    \item Isolierstoff: relative Permittivität $\varepsilon_r = 4$
    \item Leiterpositionen: $r_1 = \SI{2.5}{mm}$, $r_2 = \SI{2.3}{mm}$, $r_3 = \SI{2.1}{mm}$,
    Winkel $\varphi_{12} = 55^\circ$, $\varphi_{23} = 61^\circ$
    \item Modelllänge: $L = \SI{1}{m}$
\end{itemize}

---

\section{Geometriemodell}

Das Kabel wird als 2D-Querschnitt modelliert. Die drei Leiter werden gemäß den 
vorgegebenen Abständen und Winkeln im Inneren der kreisförmigen Abschirmung platziert.  
Der isotrope Isolierstoff füllt den Zwischenraum vollständig aus.

\begin{figure}[H]
    \centering
    \includegraphics[width=0.6\textwidth]{images/geometry.png}
    \caption{2D-Querschnittsmodell des Dreileiterkabels in COMSOL.}
\end{figure}

---

\section{Aufgabe 1: Mesh-Studie}

Ziel der Netzstudie ist die Überprüfung der numerischen Konvergenz.  
Für jedes Netzlevel wird die magnetische Energie
\[
    W_\mathrm{mag} = \frac{1}{2} \int \mathbf{B} \cdot \mathbf{H} \, dV
\]
berechnet.

Ein konvergentes Ergebnis liegt vor, wenn Änderungen von $W_\mathrm{mag}$ zwischen 
zwei Verfeinerungsstufen kleiner als 1 \% sind.

\begin{figure}[H]
    \centering
    \includegraphics[width=0.6\textwidth]{images/meshgraph.png}
    \caption{Netzstudie: Beispielnetz und Konvergenzkurve der magnetischen Energie.}
    \label{fig:mesh_study}
\end{figure}

---

\section{Aufgabe 2: Magneto-quasistatische Simulation}

Die Eigen- und gegenseitigen Induktivitäten der drei Leiter werden bei 
$f = \SI{50}{Hz}$ bestimmt. Die elektrische Isolation wird als verlustfrei angenommen, 
die Leiter besitzen jedoch einen ohmschen Widerstand.

\subsection{Theoretische Grundlagen}

Für den Strom in Leiter $i$ ergibt sich die magnetische Energie zu
\[
    W_\mathrm{mag} = \frac{1}{2} \sum_{i=1}^{3} \sum_{j=1}^{3}
    L_{ij} I_i I_j ,
\]
mit
\[
    L_{ii} = \text{Eigeninduktivität}, \qquad  
    L_{ij} = \text{gegenseitige Induktivität}.
\]

Aus der Simulation wird das magnetische Vektorpotential $\mathbf{A}$ berechnet.  
Die Induktivität ergibt sich aus
\[
    L_{ij} = \frac{1}{I_j^2} \int \mathbf{A}_i \cdot \mathbf{J}_j \, dV .
\]

Der Leiterwiderstand ergibt sich über
\[
    R = \rho \frac{L}{A},
\]
mit $\rho$: spezifischer Widerstand des Leitermaterials.

Hier Ergebnisse einfügen:

\begin{table}[H]
\centering
\begin{tabular}{c|c}
\textbf{Parameter} & \textbf{Wert} \\
\hline
$L_{11}$ & ... \\
$L_{12}$ & ... \\
$L_{13}$ & ... \\
$R_1$ & ... \\
\end{tabular}
\caption{Induktivitäten und ohmsche Widerstände.}
\end{table}

---

\section{Aufgabe 3: Elektrostatische Simulation – Kapazitäten}

Die kapazitiven Kopplungen ergeben sich aus der Lösung des elektrostatischen 
Problems. Die fundamentale Gleichung lautet:
\[
    \nabla \cdot (\varepsilon \nabla \phi) = 0.
\]

Die Ladung eines Leiters ergibt sich aus
\[
    Q_i = \int \mathbf{D}\cdot \mathbf{n}\, dA .
\]

Die Kapazitätsmatrix folgt aus
\[
    C_{ij} = \frac{Q_i}{V_j}.
\]

In COMSOL wurden die Potentialfälle jeweils so definiert, dass ein Leiter auf
\SI{1}{V} gesetzt wird, während alle anderen Leiter sowie die Abschirmung
auf \SI{0}{V} liegen. Aus den resultierenden Ladungen ergibt sich die folgende
Maxwell-Kapazitätsmatrix:

\begin{table}[H]
\centering
\caption{Maxwell-Kapazitätsmatrix des Dreileiterkabels (in \si{pF}).}
\begin{tabular}{c|cccc}
 & Leiter 1 & Leiter 2 & Leiter 3 & Schirm \\
\hline
Leiter 1 & 86.875 & -46.347 & -3.1250 & -37.403 \\
Leiter 2 & -46.347 & 115.99 & -44.687 & -24.952 \\
Leiter 3 & -3.1250 & -44.687 & 79.474 & -31.663 \\
Schirm   & -37.403 & -24.952 & -31.663 & 94.017 \\
\end{tabular}
\end{table}

Die Diagonalelemente $C_{ii}$ beschreiben die Eigenkapazitäten der Leiter unter
Berücksichtigung der Abschirmung und der gegenseitigen Kopplungen,
während die Nebendiagonalelemente $C_{ij}$ (mit $i \neq j$) die kapazitive Kopplung
zwischen den Leitern darstellen. Die letzte Zeile und Spalte geben die Kopplung
zum Schirm wieder.


\section{Aufgabe 4: Pi-Ersatzschaltbild}

Aus allen vorher bestimmten Parametern wird das äquivalente Pi-Ersatzmodell der 
Dreileiterleitung aufgebaut:

\begin{itemize}
    \item Serienparameter: $R_i$, $L_{ii}$ und induktive Kopplungen $L_{ij}$
    \item Shuntparameter: Kapazitäten $C_{ij}$ und $C_{i\text{Schirm}}$
\end{itemize}

Diagramm hier einfügen:

\begin{figure}[H]
    \centering
    \includegraphics[width=0.8\textwidth]{images/pi_ersatzbild.drawio.png} 
    \caption{Pi-Ersatzschaltbild der Leitung.}
\end{figure}

---

\section{Fazit}

In dieser Arbeit wurden alle relevanten elektromagnetischen Parameter eines 
dreipoligen Kabels durch Simulation bestimmt. Die ermittelten Größen ermöglichen 
die vollständige Beschreibung des dynamischen Verhaltens der Leitung und bilden 
die Grundlage für die Weiterverwendung im Gesamtsystemmodell.

\end{document}
