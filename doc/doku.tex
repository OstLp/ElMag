\documentclass[a4paper,12pt]{article}

\usepackage[ngerman]{babel}
\usepackage{amsmath, amssymb}
\usepackage{graphicx}
\usepackage{caption}
\usepackage{siunitx}
\usepackage{geometry}
\usepackage{float}
\usepackage{tikz}
\usepackage{booktabs}
\usepackage{hyperref}
\usepackage{csquotes}

\usetikzlibrary{circuits.ee.IEC, positioning}

\geometry{margin=2.5cm}

\title{Wissenschaftliche Dokumentation \\[4mm]
       \Large Bestimmung der Ersatzparameter eines Dreileiterkabels mittels COMSOL}
\author{Laurens Perseus, Gian Marco Näf \\ Studiengang Elektrotechnik \\ Modul: Angewandter Elektromagnetismus}
\date{\today}

\begin{document}

\maketitle

\tableofcontents
\newpage

% ------------------------------------------------------------
\section*{Aufgabenstellung}
% ------------------------------------------------------------

Die Aufgabenstellung basiert auf dem offiziellen Aufgabenblatt des Moduls
Angewandter Elektromagnetismus (Prof.~Dr.~Smajic, OST). Gefordert sind:

\begin{itemize}
    \item Durchführung einer Mesh-Studie und Bewertung der Konvergenz.
    \item Berechnung der Eigen- und Gegeninduktivitäten sowie der ohmschen Widerstände.
    \item Elektrostatische Simulation zur Bestimmung der Kapazitätsmatrix.
    \item Aufbau des Pi-Ersatzschaltbilds.
\end{itemize}

\newpage

% ------------------------------------------------------------
\section{Einleitung}
% ------------------------------------------------------------

Die elektromagnetische Charakterisierung eines mehradrigen Kabels erfordert die
Bestimmung von Induktivitäten, Widerständen und Kapazitäten über geeignete
Maxwell-basierte Modelle. Die theoretischen Grundlagen folgen der klassischen
Elektrodynamik nach Jackson \cite{jackson_classical}, Griffiths
\cite{griffiths_em} und Sadiku \cite{sadiku_fields}. 

Alle numerischen Berechnungen werden in COMSOL Multiphysics durchgeführt,
entsprechend den Dokumentationen in \cite{comsol_reference}.

Modellparameter:

\begin{itemize}
    \item Abschirmung: $r_a = \SI{5}{mm}$, $r_i = \SI{4.5}{mm}$
    \item Leiter: Radius $r_l = \SI{1}{mm}$
    \item Dielektrikum: relative Permittivität $\varepsilon_r = 4$
    \item Leiterpositionen: $r_1=2.5$ mm, $r_2=2.3$ mm, $r_3=2.1$ mm
\end{itemize}

\newpage

% ------------------------------------------------------------
\section{Geometriemodell}
% ------------------------------------------------------------

Das Modell wurde als 2D-Querschnitt aufgebaut. Der Innenraum ist vollständig mit einem
homogenen Dielektrikum gefüllt. Eine äußere Luftdomäne wurde hinzugefügt, um die
Feldabnahme korrekt abzubilden. Gemäß \cite{comsol_reference} wird empfohlen,
die Luftdomäne auf mindestens das Fünffache des Außenradius zu setzen.

Zusätzlich wurde die maximale Elementskalierung (\textit{maximum element size})
auf den dimensionslosen Wert \textit{0.2} festgelegt, um eine ausreichende
Feldauflösung an den Leitern zu gewährleisten.

Der Radius der äußeren Luftdomäne wurde auf \SI{0.01}{m} gesetzt – ein optimaler
Kompromiss aus Genauigkeit und Rechenzeit.

\begin{figure}[H]
    \centering
    \includegraphics[width=0.6\textwidth]{images/geometry.png}
    \caption{Geometrie des Dreileiterkabels.}
\end{figure}

\newpage

% ------------------------------------------------------------
\section{Aufgabe 1: Mesh-Studie}
% ------------------------------------------------------------

Die magnetische Energie dient als Konvergenzkriterium:

\[
W_\mathrm{mag} = \frac{1}{2} \int \mathbf{B} \cdot \mathbf{H}\, dV
\]
wie definiert in \cite{jackson_classical}.  

Das finale Netz ist in Abbildung \ref{fig_mesh_final} dargestellt.

\begin{figure}[H]
    \centering
    \includegraphics[width=0.65\textwidth]{images/final_mesh.png}
    \caption{Finales Netz nach Konvergenzstudie.}
    \label{fig_mesh_final}
\end{figure}

\begin{figure}[H]
    \centering
    \includegraphics[width=0.65\textwidth]{images/meshgraph.png}
    \caption{Konvergenzverlauf der magnetischen Energie.}
\end{figure}

\newpage

% ------------------------------------------------------------
\section{Aufgabe 2: Magneto-quasistatische Simulation}
% ------------------------------------------------------------

Die Induktivität zweier Leiter ergibt sich aus dem magnetischen Vektorpotential
\cite{sadiku_fields}:

\[
L_{ij} = \frac{1}{I_j^2}\int \mathbf{A}_i \cdot \mathbf{J}_j\, dV .
\]

Die Leitungswiderstände ergeben sich über das Materialgesetz:

\[
R = \rho \frac{L}{A},
\]

siehe \cite{griffiths_em}.

\subsection*{Ermittelte Induktivitäten und Widerstände}

Die aus COMSOL extrahierten Werte lauten:

\begin{table}[H]
\centering
\caption{Ermittelte Induktivitäten und Widerstände.}
\begin{tabular}{c|c}
\toprule
Parameter & Wert \\
\midrule
$L_{11}$ & $3.0933 \times 10^{-7}\,\mathrm{H}$ \\
$L_{12}$ & $3.1359 \times 10^{-7}\,\mathrm{H}$ \\
$L_{13}$ & $3.2381 \times 10^{-7}\,\mathrm{H}$ \\
$R_1$    & $6.7021 \times 10^{-5}\,\Omega$ \\
\bottomrule
\end{tabular}
\end{table}

Aus Symmetriegründen gilt:

\[
L_{12}=L_{21}, \qquad L_{13}=L_{31}.
\]

Weitere Einträge der Matrix ergeben sich analog durch Bestromung der anderen Leiter.

\subsection*{Magnetische Flussdichte}

\begin{figure}[H]
    \centering
    \includegraphics[width=0.65\textwidth]{images/magnetic_flux_density.png}
    \caption{Magnetische Flussdichte $|\mathbf{B}|$ für den Gesamtfall.}
\end{figure}

\subsection*{Magnetische Flussdichte – alle Einzelfälle}

\begin{figure}[H]
    \centering
    \includegraphics[width=0.32\textwidth]{images/B_case1.png}
    \includegraphics[width=0.32\textwidth]{images/B_case2.png}
    \includegraphics[width=0.32\textwidth]{images/B_case3.png}
    \caption{B-Feld $|\mathbf{B}|$ für Bestromung von Leiter 1, 2 und 3.}
\end{figure}

\subsection*{Stromdichte – alle Einzelfälle}

\begin{figure}[H]
    \centering
    \includegraphics[width=0.32\textwidth]{images/J_case1.png}
    \includegraphics[width=0.32\textwidth]{images/J_case2.png}
    \includegraphics[width=0.32\textwidth]{images/J_case3.png}
    \caption{Stromdichte $|\mathbf{J}|$ für Bestromung der drei Leiter.}
\end{figure}

\newpage

% ------------------------------------------------------------
\section{Aufgabe 3: Elektrostatische Simulation}
% ------------------------------------------------------------

Die elektrostatische Grundgleichung lautet \cite{griffiths_em}:

\[
\nabla \cdot (\varepsilon \nabla \phi) = 0.
\]

Die Ladung eines Leiters ergibt sich zu:

\[
Q_i = \int \mathbf{D}\cdot \mathbf{n}\, dA,
\]

woraus die Kapazitätsmatrix folgt:

\[
C_{ij} = \frac{Q_i}{V_j}.
\]

\subsection*{Potentialfeld}

\begin{figure}[H]
    \centering
    \includegraphics[width=0.65\textwidth]{images/electric_potential.png}
    \caption{Elektrisches Potentialfeld bei $V_1=1$\,V.}
\end{figure}

\subsection*{Elektrische Feldstärke}

\begin{figure}[H]
    \centering
    \includegraphics[width=0.65\textwidth]{images/electric_field.png}
    \caption{Elektrische Feldstärke $|\mathbf{E}|$.}
\end{figure}

\begin{table}[H]
\centering
\caption{Kapazitätsmatrix des Dreileiterkabels (in pF).}
\begin{tabular}{c|cccc}
 & Leiter 1 & Leiter 2 & Leiter 3 & Schirm \\
\hline
Leiter 1 & 86.875 & -46.347 & -3.1250 & -37.403 \\
Leiter 2 & -46.347 & 115.99 & -44.687 & -24.952 \\
Leiter 3 & -3.1250 & -44.687 & 79.474 & -31.663 \\
Schirm   & -37.403 & -24.952 & -31.663 & 94.017 \\
\end{tabular}
\end{table}

\newpage

% ------------------------------------------------------------
\section{Aufgabe 4: Pi-Ersatzschaltbild}
% ------------------------------------------------------------

\begin{figure}[H]
    \centering
    \includegraphics[width=0.8\textwidth]{images/pi_ersatzbild.drawio.png}
    \caption{Pi-Ersatzschaltbild des Dreileiterkabels.}
\end{figure}

\newpage

% ------------------------------------------------------------
\section{Fazit}
% ------------------------------------------------------------

Alle relevanten Ersatzparameter des Kabels wurden erfolgreich bestimmt.
Die Simulationsergebnisse zeigen die magnetischen Kopplungen zwischen den Leitern klar auf.
Durch die vollständige Bestimmung der Induktivitäts-, Kapazitäts- und Widerstandsparameter
kann das Kabel realitätsnah in Gesamtsystemsimulationen eingebettet werden.

\newpage

% ------------------------------------------------------------
\section*{Eidesstattliche Erklärung}
% ------------------------------------------------------------

Hiermit erklären wir, dass wir die vorliegende Arbeit selbstständig und ohne unerlaubte 
Hilfen erstellt haben. Alle verwendeten Quellen sind vollständig angegeben. Die Arbeit 
wurde zuvor nicht eingereicht.

\vspace{2cm}

Ort, Datum: \hrulefill

\vspace{1.5cm}

Unterschriften: \hrulefill

\newpage
\bibliographystyle{IEEEtran}
\bibliography{references}

\end{document}
