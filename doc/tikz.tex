% ---------------------------------------------------
% Sauber dargestelltes Pi-Ersatzschaltbild
% Drei Leiter + Kopplungen + Schirmkapazitäten
% ---------------------------------------------------

\begin{tikzpicture}[
    circuit ee IEC,
    thick,
    node distance=2cm,
    x=2.3cm,
    y=1.6cm
]

% =====================================================
% LEITER 1
% =====================================================
\node (L1_in) at (0,0) [contact] {};
\node (L1_R)  [right=1.5cm of L1_in] {};
\node (L1_L)  [right=1.5cm of L1_R] {};
\node (L1_out) [contact, right=1.5cm of L1_L] {};

% Serienzweig
\draw (L1_in) -- (L1_R)
      to[resistor={info=$R_1$}] (L1_L)
      to[inductor={info=$L_1$}] (L1_out);

% Shunt-Kapazitäten (gegen Schirm)
\draw (L1_in)  to[capacitor={info=$C_{1S}$}] ++(0,-1.2) node[ground]{};
\draw (L1_out) to[capacitor={info=$C_{1S}$}] ++(0,-1.2) node[ground]{};


% =====================================================
% LEITER 2
% =====================================================
\node (L2_in)  at (0,-3) [contact] {};
\node (L2_R)   [right=1.5cm of L2_in] {};
\node (L2_L)   [right=1.5cm of L2_R] {};
\node (L2_out) [contact, right=1.5cm of L2_L] {};

\draw (L2_in) -- (L2_R)
      to[resistor={info=$R_2$}] (L2_L)
      to[inductor={info=$L_2$}] (L2_out);

\draw (L2_in)  to[capacitor={info=$C_{2S}$}] ++(0,-1.2) node[ground]{};
\draw (L2_out) to[capacitor={info=$C_{2S}$}] ++(0,-1.2) node[ground]{};


% =====================================================
% LEITER 3
% =====================================================
\node (L3_in)  at (0,-6) [contact] {};
\node (L3_R)   [right=1.5cm of L3_in] {};
\node (L3_L)   [right=1.5cm of L3_R] {};
\node (L3_out) [contact, right=1.5cm of L3_L] {};

\draw (L3_in) -- (L3_R)
      to[resistor={info=$R_3$}] (L3_L)
      to[inductor={info=$L_3$}] (L3_out);

\draw (L3_in)  to[capacitor={info=$C_{3S}$}] ++(0,-1.2) node[ground]{};
\draw (L3_out) to[capacitor={info=$C_{3S}$}] ++(0,-1.2) node[ground]{};


% =====================================================
% UNTERE KNOTEN FÜR KOPPLUNGEN
% (Schön nach links versetzt für klare Darstellung)
% =====================================================
\node (N12) at (-2,-1.5) {};
\node (N23) at (-2,-4.5) {};
\node (N13) at ( 2,-3 ) {};

% =====================================================
% Gegenseitige Kapazitäten C_ij
% =====================================================

% C12
\draw (L1_in) -- ++(-1,0) to[capacitor={info=$C_{12}$}] (L2_in);

% C23
\draw (L2_in) -- ++(-1,0) to[capacitor={info=$C_{23}$}] (L3_in);

% C13
\draw (L1_in) -- ++(0.8,0) to[capacitor={info=$C_{13}$}] (L3_in);


% =====================================================
% Gegenseitige Induktivitäten M_ij (gestrichelt)
% =====================================================

\draw[dashed] (L1_L) -- node[right] {$M_{12}$} (L2_L);
\draw[dashed] (L2_L) -- node[right] {$M_{23}$} (L3_L);
\draw[dashed] (L1_L) -- node[left] {$M_{13}$} (L3_L);


% Beschriftungen
\node[left]  at (L1_in)  {L1 Eingang};
\node[right] at (L1_out) {L1 Ausgang};

\node[left]  at (L2_in)  {L2 Eingang};
\node[right] at (L2_out) {L2 Ausgang};

\node[left]  at (L3_in)  {L3 Eingang};
\node[right] at (L3_out) {L3 Ausgang};

\end{tikzpicture}